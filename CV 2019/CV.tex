\documentclass{my_cv}

\usepackage[dvipsnames]{xcolor}

\usepackage{geometry}
\geometry{a4paper, total={7in,9in}}


\begin{document}

\name{Aviral \textbf{Janveja}}
\contact{Peter-hille-Weg 13}{Padeborn}{Germany 33098}{aviraljanveja@mail.com}{(+49)-1776926107}


\section{\textcolor{BrickRed}{Dir}ection}
My Research Interests include Mathematical Logic, Algorithms, Complexity theory, Quantum Computing and Artificial Intelligence. I like combining research with innovation to help solve challenging problems.\\
As Prof. Richard Feynman put it beautifully, "The \emph{worthwhile} problems are the ones you can really solve or help solve, the ones you can really contribute something to."

\section{\textcolor{BrickRed}{Edu}cation}
\datedsubsection{University of Paderborn}{2018--Present}
Master of Science in Computer Science.
\datedsubsection{SRM University}{2012--16}
Bachelor of Technology in Information Technology.\\
Thesis: Design and Implementation of heap data structure in A*\\ algorithm for path finding in computer games.
\datedsubsection{J.H.Ambani School}{2009--12}
CBSE Senior School: CGPA = 9.07/10 (German Equivalent - 1.3)\\
CBSE Secondary School: CGPA = 10/10 (German Equivalent - 1.0)

\section{\textcolor{BrickRed}{Res}earch}
\datedsubsection{University Seminar: Machine Learning}{2019--Present}
Preparing a Seminar Talk on the paper "A Semantic Loss Function for Deep Learning with Symbolic Knowledge" from ICML-2018. To be given to prof. Eyke Hüllermeier in the Intelligent Systems \& Machine Learning department, University of Paderborn.

\section{\textcolor{BrickRed}{Pro}jects}
\datedsubsection{University Project-Group: NICE-IDEA}{2019--Present}
\projectitems{We develop an educational application for school children based on 'Pentomino' logic puzzle. We made use of Godot game-engine, utilizing its own optimized GDScript language.}{We also perform inter-disciplinary research on learning behaviors, human computer interaction techniques and analysis of eye-tracking data.}
\datedsubsection{Pocket/Perceptron Algorithm for Hand-Written Character\\ Recognition}{2019--Present}
\projectitems{Implementing the Pocket algorithm on MNIST data-set for hand-written character classification.}{Using Pocket strategy to enable classification through Perceptrons over linearly inseparable data.}

\section{\textcolor{BrickRed}{Emp}loyment}
\datedsubsection{BYJU's The Learning App}{2016--17}
Business Development Associate
\projectitems{Worked on Customer-Strategy and Marketing.}{Coordinated Supply-Chain.}

\section{\textcolor{BrickRed}{Ski}lls}
Design and Analysis of Algorithms, Data-Structures, Programming Fundamentals, General Aptitude \& Language.
\projectitems{Java, C++, Python, GD-Script, Command-Line, Git, \LaTeX, DBMS-SQL.}{GRE - 313/340 (2017), TOEFL iBT - 101/120 (2017), German A1 - Grade 1.7(88\%)(Learning A2).}

\section{\textcolor{BrickRed}{Ach}ievements}
\projectitems{Central Board of Secondary Education, India Top 1\% Student.}{Inspire Internship at National Institute of Technology, Surat, India.}
\projectitems{Robotics Event Organizer at Tech-Fest: Aaruush 2014, SRM University.}{Tennis: National-Level, Football: State-Level, Karate: State-Level.}

\textbf{\today.}

\end{document}